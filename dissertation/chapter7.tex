\graphicspath{{../images/ch1/}}	% Image directory


\chapter{Conclusion}
\label{chap:conclusion}

In this dissertation I discussed several projects I worked on as part of my PhD work. Although the projects are in seemingly disparate areas---carbon nanotube studies, hybrid insulator-conductor nanopores, and several different resistive pulse applications, they are all united in that they concern the ionic conductance properties of channels. To narrow down their description, the first two projects, carbon nanotubes and the aforementioned hybrid nanopores, focus on the conductance properties of teh channels themselves. Such devices might see application as elements in novel ionic circuits. The last three projects are not directly concerned with the conductance properties of ion channels, but rather the \textit{change} in their ionic conductance properties when interfered with by a passing particle, which can then be studied to understand the properties of the particles themselves rather than the channels that allow their passage. However, we note that fundamentally, all projects studied in this dissertation involve the physics of ion conductance.

In the next few sections, we'll quickly recap each of the sections of this thesis.

\section{Introduction}
	
	In the introduction we reviewed all of the requisite knowledge for understanding each of the projects in the body of this thesis. Each of these projects ultimately involves ion transport in conducting channels. The physics involved includes electrostatics, statistical mechanics, and fluid mechanics. More specifically, we discussed the Nernst-Planck equations, that describe the motions of ions in an electric field and moving fluid medium. We also discussed the motion of the fluid medium itself, described by the famous Navier-Stokes equations. In order to understand the nanoscale effects that give nanopores their interesting conductance properties, we introduced the Poisson-Boltzmann equation, which predicts the formation of the electrical double layer, a small screening layer of counterions present within a small distance of any charged surface in a solution. Some of the consequences of the formation of the electrical double layer were discussed, including surface conductance, electroosmosis, and electrophoresis, the latter two which were the two main drivers of particle transport in nanoscale resistive pulse experiments. Finally, we introduced resistive pulse experiments and explained how the change in the conductance properties of pores by a passing particle can be used to study the properties of the particle itself.

\section{Ion conductance of nanopores}

	\subsection{Ion conductance of single carbon nanotubes}
	
		In the first chapter of the body of this thesis we discussed work done on studying and characterizing a new carbon nanotube single nanopore platform. The platform was developed by collaborators at Lawrence Livermore National Laboratory, and devices were tested in the Siwy lab. Carbon nanotubes are, at hte time of the writing of this thesis, a relatively poorly understood nanopore platform. They exhibit novel conductance properties unmatched or seldom seen in other pores, including ultrafast flow of water, ion selectivity despite not having native surface charges, unexpetedly large conductance properties (probably due to Grotthus conduction of protons), and a sublinear conductance-concentration relationship (most pores have piecewise conductances, with a transition from a saturated region at low concentration to a linear conductance region). However, although there seems to be a consensus in the literature that carbon nanotube porins demonstrate these properties, there is very little overlap in the numerical reports coming from different groups, for instance in the exponent in the carbon nanotube's conductance power law or even in the order of magnitude of theirconductance values at a given concentration! Furthermore, it is not immediately apparent that all of the results reported accurate conductance through a single pore; guaranteeing a leak-free passage through a carbon nanotube which may have a length orders of magnitude greater than its diameter is not easy. Therefore, in order for their true behavior to be better pinpointed, it is necessary to have a collaborative effort where many different carbon nanotube platforms are studied in parallel. If many of the observed traits persist throughout different pore support architectures, it is likely that the true behavior of the device has been honed in on. Our efforts were therefore to promote a newly conceived architecture for carbon nanotube studies and reports its conductance properties. Our devices were characterized by a power law in their conductance-concentratoin relationship, large currents that are orders of magnitude larger htan expected for a classical device of hte same shape and size, and even were shown to potentially allow for the passage of single molecules such as DNA, which could be useful in resistive pulse experiments. While the results are promising, additional tests must be performed on this platform in order to verify hte observations and ensure that leakage pathways did not contribute to the pore's properties.
		  
	\subsection{Ion conductance of hybrid insulating-conducting nanopores}
		
		The vast majority of nanopores studied are entirely insulating. However, it is conceivable that instead metallic, conducting nanopores could be used instead. As far as electrostatics is concerned, the difference between insulating and conducting bodies is in the electrostatic boundary conditions imposed at their surfaces; in metals, external voltage gradients must be cancelled to ensure no electric fields are present in the body of the conductor. The net result is that free charges in the metal must rearrange themselves in order to cancel the field, and in doing so situate themselves at the surface of the metal, and act just like static charges present at any other surface in the system. We developed a hybrid insulator-metallic nanopore system by depositing a thin layer of gold on top of a silicon nitride membrane, and drilled nanopores through the combined layers to create a new nanopore system. We studied the pore's properties by measuring their current-voltage characteristics, and found that the system behaved as a rectifying diode, with a large conductance for one voltage polarity and a near-zero conductance at the opposite voltage polarity. Furthermore, we found that the direction of hte rectification could not be explained by static charges present on the surface of the silicon nitride. With the aid of COMSOL simulations, we showed that the rectification could be explained if free charges in the metal strongly polarized the surface in response to the external voltages. In the `on' state (high conductance state), the total charge configuration led to enhanced ion sourcing at both ends of the pore, and a conductance value nominally higher than expected for a completely neutral pore. However, in the `off' state (low conductance), a depletion zone is formed at the junction between the silicon nitride insulator and the gold conductor. This depletion zone leads to a very low conductance in the pore. The results are important for two reasons. First, the pores exhibit novel conductance on their own and could be used as diode elements in ionic circuits. Second, the physics and chemistry of induced charges on conductors present in solution have been seldom studied, and this result confirms that these charges and contribute to the conductance properties of the device just as static charges do.
		
	\subsection{Resistive pulse studies of mesoscale rods}
	
		In chapter \ref{chap:rods}, we explained how resistive pulse sensing can be extended to measure not only the volume of a passing particle, but the length of the passing particle as well. Particles measure the local pore radius as they translocate, and therefore we can view their resistive pulse amplitudes as a `map' of the local pore interior. When particles with displacement lengths much shorter than the characteristic length scale of diameter undulations in the pore translocate, they map the pore interior with high resolution. Oppositely, particles with long length displacements map the pore interior with low resolution. The resistive pulse signals of these long particles can be represented as a moving average of the resistive pulse signals of shorter particles. We make use of this fact to show that by performing moving averages of the signals of short particles for a variety of widths, we can compare the resulting transformed signal with the raw signal of the long particle. The moving average length for which the two signals are most similar is identified as the length of the unknown long particle. In order to compare the similarity of the two resistive pulses, we used a computational algorithm known as dynamic time warping. To test the predescribed principles, we performed experiments with polystyrene spheres, short silicon rods, and long silicon rods. We showed that at a qualitative level the resistive pulse signals of the short rods and long rods were very different, and furthermore that small scale features in the signals of the shorter rods were not present in the signals of the longer rods, in accordance with the previously described model. We employed dynamic time warping to make a quantitative comparison, and found that the distribution of the lengths of the long rods could be somewhat accurately recovered using this method. Finally, in the course of running the experiments with short rods we discovered the scale of their resistive pulse amplitudes was far larger than expected for their volumes. The discrepancy between expected resistive pulse amplitude and observed resistive pulse amplitude was as large as a factor of two. In order to explain this result, we considered a model where the rods were rotating extremely rapidly in the solution. This type of rapid rotation could `stir' the local ion solution in their vicinity, reducing its effective conductivity. Furthermore, we argued that this type of rotation could be physically realizable due to the random undulations in the channel which would create non-uniform electric fields, and therefore induce a large torque on the rods.
		
	\subsection{Hybrid resistive pulse-optical platform}
	
		Many equations exist for predicting the resistive pulse amplitudes of various particles, however the vast majority of these equations are only approximations that do not take into account geometry-dependent effects present in the channel. For instance, the equations for RP amplitude all assume channels with lengths much greater htan their diameter, eliminating the confounding effects of the channel's entrance and exit. We argued that these effects may be important for certain channel geometries, and that oftne the ideal geometries used to derive analytical solutions may not always be realizable. In order to better understand the confounding position and geometry dependent effects, we devised an experimental platform using transparent planar microfluidic channels that were simultaneously measured with resistive uplse signals and optically using a high-speed camera. Polystyrene bead suspensions were driven through the channel with a syringe, while the resistive uplse signal and camera data was simultaneously recorded. After the data was recorded, the two independent data streams were synchronized so that every frame in the camera data matched with a single data point in the resistive pulse signal corresponding to the same instant in time. The matched signals were then used to create `resistance maps' of the channels, a two-dimensional color map where the color of each data point corresponds to the resistive pulse amplitude at that exact location. The resistance maps were used to explore positional effects of the resistive uplse amplitudes; for instance, we were able to directly show that translocation of particles off-axis yielded larger resistive uplse amplitudes, and that the resistive uplse amplitude does not attain its maximal value until the particle is well inside the channel. Additionally, we used the resistance map to study the positional dependence of resistive uplse amplitudes in channels with non-constant widths. In order to perform the previously mentioned steps, a library for analyzing hte hybrid resistive uplse optical measurements was developed.
		
	\subsection{Particle deformability cytometry using resistive pulse sensing}
	
		Continuing hte overarching theme of the studies on extending the resistive pulse technique, in chapter \ref{chap:cells}, we explained how channels with central cavities can induce repeatable deformations of non-rigid particles, and developed a model for how the deformations would occur and hte effects htey would cuase on the resistive uplse signals. We presented experiments performed on two different cell types in channels of vairous geometries. We showed how the observed deformations agreed with the deformation predicted by our model, and furthermore, provided results from finite element simulation performed using COMSOL that confirm the model and experimental results. At the time of the writing of this thesis, the obtained results are still preliminary and the platform's development is ongoing. We briefly discussed the hurdles that would be necessary to overcome in order to realize a device capable of high-throughput cell deformation measurements using the resistive pulse signal alone. 
	
		




			
			
			

			
			
			
			

			
			


			
	





%%% Local Variables: ***
%%% mode: latex ***
%%% TeX-master: "thesis.tex" ***
%%% End: ***
