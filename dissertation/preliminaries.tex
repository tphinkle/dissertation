\thesistitle{Applications of Synthetic Microchannel and Nanopore systems}

\degreename{Doctor of Philosophy}

% Use the wording given in the official list of degrees awarded by UCI:
% http://www.rgs.uci.edu/grad/academic/degrees_offered.htm
\degreefield{Physics}

% Your name as it appears on official UCI records.
\authorname{Preston Hinkle}

% Use the full name of each committee member.
\committeechair{Dr. Zuzanna S. Siwy}
\othercommitteemembers
{
  Dr. Jun Allard \\
  Dr. Ilya Krivorotov
}

\degreeyear{2017}

\copyrightdeclaration
{
  {\copyright} {\Degreeyear} \Authorname
}

% If you have previously published parts of your manuscript, you must list the
% copyright holders; see Section 3.2 of the UCI Thesis and Dissertation Manual.
% Otherwise, this section may be omitted.
% \prepublishedcopyrightdeclaration
% {
% 	Chapter 4 {\copyright} 2003 Springer-Verlag \\
% 	Portion of Chapter 5 {\copyright} 1999 John Wiley \& Sons, Inc. \\
% 	All other materials {\copyright} {\Degreeyear} \Authorname
% }

% The dedication page is optional.
\dedications
{
	
}

\acknowledgments
{
	
}


% Some custom commands for your list of publications and software.
\newcommand{\mypubentry}[3]{
  \begin{tabular*}{1\textwidth}{@{\extracolsep{\fill}}p{4.5in}r}
    \textbf{#1} & \textbf{#2} \\ 
    \multicolumn{2}{@{\extracolsep{\fill}}p{.95\textwidth}}{#3}\vspace{6pt} \\
  \end{tabular*}
}
\newcommand{\mysoftentry}[3]{
  \begin{tabular*}{1\textwidth}{@{\extracolsep{\fill}}lr}
    \textbf{#1} & \url{#2} \\
    \multicolumn{2}{@{\extracolsep{\fill}}p{.95\textwidth}}
    {\emph{#3}}\vspace{-6pt} \\
  \end{tabular*}
}

% Include, at minimum, a listing of your degrees and educational
% achievements with dates and the school where the degrees were
% earned. This should include the degree currently being
% attained. Other than that it's mostly up to you what to include here
% and how to format it, below is just an example.
\curriculumvitae
{

\textbf{EDUCATION}
  
  \begin{tabular*}{1\textwidth}{@{\extracolsep{\fill}}lr}
    \textbf{Doctor of Philosophy in Physics} & \textbf{2017} \\
    \vspace{6pt}
    University of California, Irvine & \emph{Irvine, California} \\
    \textbf{Bachelor of Science in Physics} & \textbf{2011} \\
    \vspace{6pt}
    The Ohio State University & \emph{Columbus, Ohio} \\
  \end{tabular*}

\vspace{12pt}
\textbf{RESEARCH EXPERIENCE}

  \begin{tabular*}{1\textwidth}{@{\extracolsep{\fill}}lr}
    \textbf{Graduate Student Researcher} & \textbf{2012--2017} \\
    \vspace{6pt}
    University of California, Irvine & \emph{Irvine, California} \\
  \end{tabular*}

\vspace{12pt}
\textbf{TEACHING EXPERIENCE}

  \begin{tabular*}{1\textwidth}{@{\extracolsep{\fill}}lr}
    \textbf{Teaching Assistant} & \textbf{2012--2016} \\
    \vspace{6pt}
    University of California, Irvine & \emph{Irvine, California} \\
  \end{tabular*}
  
   \begin{tabular*}{1\textwidth}{@{\extracolsep{\fill}}lr}
    \textbf{Teaching Assistant} & \textbf{2012} \\
    \vspace{6pt}
    The Ohio State University & \emph{Columbus, Ohio} \\
  \end{tabular*}


  
\pagebreak

\textbf{REFEREED JOURNAL PUBLICATIONS}

	\bibentry{Hinkle2017}
	\bibentry{Yang2016}
	\bibentry{Qiu2015}

% \vspace{12pt}
% \textbf{REFEREED CONFERENCE PUBLICATIONS}
% 
% 	\mypubentry{Awesome paper}{Jun 2011}{Conference name}
% 	\mypubentry{Another awesome paper}{Aug 2012}{Conference name}

\vspace{12pt}
\textbf{SOFTWARE}

  \mysoftentry{nanoIV}{https://github.com/tphinkle/nanoIV}
  {Keithley 6487 Picoammeter control GUI program. Allows measurement of current-voltage and time-series data.}
  
  \mysoftentry{pore stats}{https://github.com/tphinkle/pore_stats}
  {GUI program and Python library for extracting and analyzing resistive pulse data.}
  
  
}

% The abstract should not be over 350 words, although that's
% supposedly somewhat of a soft constraint.
\thesisabstract
{
	There are a diverse range of applications involving fluidic systems at the micro- and nanoscale. Making use of the nanoscale physics that takes place in the vicinity of charged surfaces, there is the possibility that nanopores, holes on the order of $\SI{1}{nm}$ in size, could be used to make complex integrated ionic circuits. For inspiration on what such circuits could achieve we only need to look to biology systems, immensely complex machines that at their most basic level require precise control of ions and intercellular electric potentials to function. In order to contribute to the ever expanding field of nanopore technology, we engineered novel hybrid insulator-conductor nanopores that behave analagously to ionic diodes, which allow passage of current flow in one direction but severely limit the current in the opposite direction. Not only did the behavior of the pore give key insights into the fundamental physics and chemistry underlying the electrical double layer, but it can also be considered as a standard rectifying element in ionic circuits. Another application of ion conducting channels is a particle sensing method, known as resistive pulse sensing. We present three main experiments that expand the capacity of resistive pulse sensing applications. First, we demonstrate how resistive pulse sensing in pores with longitudinal irregularities can be used to measure the lengths of individual nanoparticles. Then, we describe an entirely new hyridized method, whereby resistive pulse sensing is combined with optical imaging. The hybrid method allows for validation of the resistive pulse signals, and will greatly contribute to their interpretability. We present experiments that explore some of the possibilities of the hybrid method. Then, building off the hybrid method we present our findings for experiments performed with using resistive pulse sensing to measure the deformability of particles. Using a novel microfluidic channel design, we were able to reproducibily induce deformation of cells in a controlled manner. We describe how these deformations could be detected with the resistive pulse signal, paving the way for resistive pulse sensing based cell deformability cytometers.
}


%%% Local Variables: ***
%%% mode: latex ***
%%% TeX-master: "thesis.tex" ***
%%% End: ***
