\documentclass[a4paper,10pt]{article}
\usepackage[utf8]{inputenc}
\usepackage{amsthm}
\usepackage{siunitx}

\theoremstyle{definition}
\newtheorem{definition}{Definition}[section]
 
\theoremstyle{remark}
\newtheorem*{remark}{Remark}


%opening
\title{Applications of Synthetic Microchannel and Nanopore systems}
\author{Thomas Preston Hinkle}

\begin{document}

\maketitle

\begin{abstract}
Abstract

\end{abstract}


 


\section{Introduction}

The introduction to this dissertation is structured as follows. First, we introduce the main topic of this dissertation and of my PhD research work, the nanopore. This section serves to introduce the reader first, to the basic definition of a nanopore and then to some of their applications, in addition to miscellaneous facts about nanopores that provide context for why they are of interest to scientists. This section is entirely qualitative and does not aim to make any statements about the underlying physics needed to understand nanopore systems. Subsequently, I introduce in three stages the physics that is relevant to nanopore systems. It turns out that if we take a macroscaled system (say, a tube rather than a nanotube) and slowly shrink it in size, then as the system size shrinks new types of physics emerges and become increasingly relevant to the system. We will start by addressing the relevant physics at the milliscale and shrink in intervals of powers of three, stopping at the nanoscale, and briefly discussing physics at the sub-nanoscale. Therefore, the progression looks like milli-, micro-, nano-, and finally, sub-nano-. To provide a bigger picture of the journey through different length scales, the progression looks like ant, cell, protein, molecule.

\subsection{Nanopores}

Unlike most physics jargon, the name nanopore is entirely self-explanatory. In fact, the following is a terse (but accurate) definition of a nanopore:

\theoremstyle{definition}
\begin{definition}{Nanopore}
A nanopore is a hole on the order of $\SI{1}{nm}$ in size
\label{def:1}
\end{definition}

As we will discover, this definition, while accurate, does not do justice to the field of nanopore research. 

Nanopores generally belong to two classes, biological and synthetic. Because nanopores can be found in biology, many scientists are interested in studying them in their own right, to better understand how biological systems function. On the other hand, nanopores, both biological and synthetic, can be used in a surprisingly large variety of applications, with the total count of such applications constantly increasing. In this section we will briefly discuss the types of biological nanopores and where they are found in nature, and then move on to describing different types of synthetic nanopores. Given the loose criteria required by \ref{def:1} for something to constitute a nanopore, listing off every type of either biological or synthetic nanopores would be impossible; instead, I seek to introduce the most relevant types, with an admitted bias towards the types of nanopores I am most familiar with and that I have worked with in my research. After discussing the various types of nanopores, I will give a summary of applications in which nanopores see use. Some of these applications, for instance resistive-pulse sensing, are a major part of this dissertation and will therefore be discussed in much greater detail later on, while others, such as DNA sequencing, are important enough to be mentioned but not discussed in significant detail later on. An extensive list of references, including those not explicitly referenced in this work, will be included should the reader wish to read more on a particular topic.

\subsubsection{Biological nanopores}

As mentioned before, nanopores exist naturally in biology. Their most important function is to enable passage from one region to another, much like a tunnel does; however, where a tunnel permits cars, trucks, etc., the passengers in nanopore systems are water, ions, proteins, etc. Nanopores typically connect regions that are necessarily divided in order for a cell to function. Membrane-bound nanopores are structures made out of protein that are embedded in a cell's lipid bilayer that connect the outside of a cell to the inside of a cell. They serve the useful function of regulation osmotic pressure by allowing water to enter and exit the cell, and also balance voltage differences by allowing passage of important electrolytes such as sodium, potassium, and chloride. Therefore, maintaining homeostasis is one of the chief functions of these types of nanopores. These pores also enable passage of messenger molecules that enable extracellular messenging, crucial to all multicelled organisms. Like lipid bilayer pores, there are also pores in the nuclear membrane that enable transport into and out of the nucleus of a cell. These pores' chief responsibility is to allow the passage of messenger molecules in and out of the nucleus. 

The most fascinating property of nanopores is their capacity for active management of transport. For instance, some nanopores may open or close when the voltage difference across them surpasses a threshold; other types of nanopores are \textit{ion selective}, permitting passage of some types of ions while denying passage to others. \textit{Active} transport is absolutely crucial to biological systems; if all nanopores were passive holes that permitted anything to pass through, the system would equilibrate in a homogeneous, non-functional soup of maximum entropy---life simply could not exist. Fortunately for us, nanopores are active transport regulators and life goes on. Going back to our tunnel analogy, nanopores act more like regulated tunnels, for instance some with guards that permit small cars only and deny trucks, or permit passage in only one direction, etc.

\subsubsection{Synthetic nanopores}

Advances in nano- and microfabrication have led to an enormous number of different types of synthetic nanopores. Synthetic nanopores can be made from a variety of different types of materials, and often their geometry (shape and size), as well as their surface chemistry properties, can be tailored to introduce specific types of behavior. However, the type of material used to make the nanopore most often determines the types of geometries and chemistries that are allowed. The following is a list of common types of nanopores, with a description of the type of material they are made from, their permissable length scales, and other miscellaneous relevant facts about them.

\textbf{Monolayer pores.} The invention of graphene, a monolayer of carbon atoms, and later MoS2, opened the possibility of creating nanopores with a length of a single atom. These types of pores are created by punching through the thin monolayer, typically with an electron or ion beam. While still a very new field of study, these pores may see future use in desalination (removing electrolyte ions from water).

\textbf{Carbon nanotubes.} A carbon nanotube can be conceptually understood as a rolled up tube of one or a stack of graphene layers. Graphene is a a monolayer of carbon atoms, and therefore a carbon nanotube itself is a type of pure crystal structure. The lattice arrangement of carbon atoms permits only certain tube diameters, which itself depends on the chirality of the tube (the way the tube is rolled up). Carbon nanotubes are interesting to researchers because of the exotic behaviors they exhibit that are not present in most other types of nanopores---for instance, they exhibit frictionless transport of water, hydrogen-dominant conductance (\textit{via} the Grotthus mechanism), ion selectivity, and more. As of the submission of this dissertation, the use of CNTs as nanopore is still a new field of study, and the physical mechanism for the previously mentioned behaviors is still poorly understood. Nevertheless, the extreme confinement ($\SI{1}{nm}$ and below for small CNTs) and the atomic-level precision of their structure are believed to be responsible. Another interesting aspect of the smallest CNTs is the breakdown of mean-field physics in describing their behavior. For instance, the Navier-Stokes equations which describe fluid dynamics, breaks down at this level because the water must be considered at the molecular level; the mean-field approach is no longer valid. Because they are still new, we do not yet know the exact applications that these nanopores will see use in. However, it is likely that they could be used in desalination applications (removing ions from water), or in ionic circuits as cation selective elements.

\textbf{Silicon nitride.} Silicon nitride is a type of crystalline semiconductor that permits engineering of nanopores through several different fabrication processes. Briefly, very thin layers of silicon nitride (e.g.~$10-100$ nm) are grown, and a hole is bored through via either electron or ion beam milling. For instance, in one project of this dissertation I describe a project conducted with silicon nitride pores that were drilled via high-energy transmission electron microscope (TEM). Depending on the type of mill used (ion or electron), the size of these nanopores ranges, but the diameter is generally in the range of $1-100$ nm. One advantage of silicon nitride pores is their smooth interior geometries, as well as the native silane chemistry on the surface that permits many types of chemical modifications. These types of pores are used as ionic rectifiers (after modification), and in resistive pulse applications.

\textbf{Glass nanopipettes.} Quartz pipettes can be heated and slowly stretched, reducing the tip diameter with the possibility of reaching the nanoscale. Unlike the previously mentioned pores which all had an approximately cylindrical geometry, these nanopipettes have a conical shape. These pores have a couple advantages. First, the stretching process itself is simple and can be used to create many pores over a short amount of time. Second, the surface chemistry of the glass or quartz is amenable towards chemical modification. These pores may be used in ionic circuits, in resistive pulse sensing, or as a surface probe, by monitoring the current through the pipette as it approaches a charged surface.

\textbf{Track-etched polymer nanopores.} These types of pores are perhaps the most robust, dependable types of nanopores. Pore formation is a multistep process. First, untouched polymer membranes are irradiated by a single heavy isotope of an element such as gold and xenon, that has been accelerated to high speeds in a particle accelerator. The ion rips through the membrane, uniformly dispersing some of its kinetic energy into the surrounding polymer, and breaking the bonds in the polymer surrounding its trajectory. This location of damaged polymer bonds is known as the `damage track'. An ion detector is placed at the exit point of the membranes so that the beam can be switched off when a single ion is detected. This detection, along with a solid mask placed in front of the beam that blocks the vast majority of the ions in the beam, ensures that films that only have a single damage track can be prepared. This step, known as irradiation, must be performed off-site at a particle accelerator. After heavy ion irradiation, the membranes are immersed in an etchant solution such as NaOH or KOH. The etchant preferentially attacks the polymer in the damage track, clearing out the polymers along the track much faster than elsewhere in the membrane. Once the track has been etched out, the rest of the membrane is isotropically etched slowly by the NaOH. Pores prepared by the track-etch technique have a few advantages, including a customizable size and geometry. By putting etching solution on only one side of the membrane, it is possible to create conical pores of various aspect ratios. By applying a voltage across the pore during the etching process, the ionic current can be monitored and the etching process can be stopped at a particular current level, allowing the researcher customize the pore diameter. Another advantage of these pores is the carboxyl groups native to their surfaces, which permit many types of useful chemical modifications.

\textbf{Hybrid biological-synthetic nanopores.} Biological nanopores, such as alpha hemolysin, aerolysin, or MSPA to name a few, can be isolated from their host biological systems and inserted into synthetic systems, forming a hybrid biological-synthetic complex. In these cases, the pore is entirely biological, but the complex does not appear naturally in nature, and the pores may be used in engineering or scientific applications. To make matters even more confusing, such pores may also be genetically modified to change some of their behavior, meaning they are derived from biology but are synthesized in the lab. For instance, the company Oxford nanopore invented a genetically modified alpha hemolysin nanopore that is especially adept at differentiating between nucleotides, and is currently being used in DNA sequencing applications.

\subsubsection{Applications (introductions)}

So far, I have hinted at or referred to nanopore applications without getting into specifics. Just as there are too many types of nanopores to enumerate, there are too many applications of nanopores to list them all off. This is a list of some of the most important applications of nanopores.

\textbf{Analyte detection and characterization with the resistive pulse technique.} Surprisingly, nanopores may be used as a sort of particle characterizer using something called the resistive pulse technique, which works as follows. A voltage is applied across the nanopore, which induces a measurable ionic current to flow through the channel. If a particle enters the nanopore, the particle will occupy a volume that otherwise would be occupied by high-conductivity ions, increasing the pore's resistance, and decreasing the measured current. It turns out that by studying the nature of the decrease in the current, we can gather some information about the transiting particle. For instance, by counting the number of pulses in the measured ionic current, we can determine the concentration of particles in the suspension. The size of the particle can be determined by relating it to the depth of the pulse---larger particles block the current to a larger degree, and therefore create deeper resistive pulses. If the particle is at least partially driven through the channel \textit{via} electrophoresis (discussed later), then the particle's surface charge can be determined by the length of the pulse, or `dwell time' of the particle.

One advantage of resistive pulse sensing is its scale-independence, which is a result of the generality of the physics involved. For this reason, resistive pulse sensing has been used in a large number of applications. The original use of the resistive pulse technique was in performing red blood cell counts (channel size approximately $\sim\SI{100}{\mu m}$ which was achieved by Coulter in 1953; this is the reason why sometimes the resistive pulse technique is referred to as the Coulter counter priniciple, or simply the Coulter principle. Since Coulter's original design, the resistive pulse technique has been used in counting smaller specimens such as exosomes, proteins, and viruses, used to measure particle rigidity, and perhaps most importantly, in DNA sequencing. The basic idea of resistive-pulse sensing for DNA sequencing is that the four types of nucleotides have slightly different sizes, and therefore lead to four unique current blockages. DNA can be slowly threaded through a nanopore while the current is monitored, and the discrete states in the resulting time-series of the measured current yield the sequence of DNA.

\textbf{Water desalination.} Charged nanopores are slightly ion selective, meaning they preferentially allow one polarity of charge over another to pass through. If one side of a nanopore membrane is filled with electrolyte solution and a pressure is applied, water will pass through the pore, and due to the selectivity of the membrane, fewer ions will pass through than are contained in the bulk. In this way, nanopores can be used for water desalination. This mechanism of salt rejection is electrostatic, since the finite potential in the vicinity of the pore walls is responsible for screening ions. Traditional water desalination relies on reverse osmosis membranes that are completely impermeable to salts, and therefore \textit{sterically} reject ions.  In future sections we will discuss the physics involved with water desalination.

\textbf{Nanopore ionics and ionic circuits.} Ion transport is fundamentally important to all cellular life. At the most basic level, single cells maintain homeostasis with their environment by maintaining a careful balance of salt ions across their membranes. This balance is regulated by a rich variety of ion channels, each with their own specific functionality. For instance, potassium can diffuse through potassium channels at nearly the bulk diffusion rate (i.e. the channel hardly impedes their motion), while sodium is completely blocked, an amazing feat considering the ions are nearly identical, save a minute difference in their hydration radii. Beyond maintaining homeostasis, ion channels are also responsible for propagating electric signals along axons, crucial for signaling in multicellular organisms. 

One of the holy grails in nanopores is being able to reproduce the functionality of these biological pores, that were created by millions or billions of years of evolution, in synthetic systems. The ability to do so could create a revolution in ion conducting systems, analagous to the revolution in electronics that was created by the invention of the solid-state transistor. However, the goal of ionics is not to replace wires, capacitors, and transistors with ion channels and pores; indeed, ions travel orders of magnitude slower than electrons, and therefore ionic systems are expected to act more slowly than electronic systems. Instead, integrated ionic circuits could be used for complex signaling since each ion carries its own information, and for controlling the transport of biomolecules. One needs only to look at biology for inspiration on the application of complex ion channels.

\textbf{Nanomaterials science.}

\subsection{Nanopore science---physics and chemistry}

The actors at play in nanopore systems include the fluid solvent that fills the pore and its exterior regions, the charged ion species, the pore surfaces themselves, the electrodes, and finally, particles present in the system. In the following sections we will delve into the physics that determines the interplay between these actors. As it turns out, as we shrink a system size from the macro to nanoscale, new regimes of physics become increasingly important. Therefore, organizationally we will cover the physics that is most relevant in macrosystems and work our way down to the nanoscale, stopping at the landmark scales where new important physics becomes emergent. To give a brief preview of what is to come, the most important physics involved in nanopore theory is fluid mechnaics (to describe the forces on the solvent), electrostatics (because of the voltage boundary conditions imposed by electrodes), electrokinetics (due to migration of charged species under electric fields), and statistical mechanics (to describe the ensemble behavior of the ions). Before moving on, it is important to note that concepts in chemistry are also very important in nanopores. For instance, the physical chemistry is entirely responsible for explaining the mechanism for how an ion current turns into an electron current at the electrode-liquid interface. As another example, the charges present at the surfaces in almost all surfaces in contact with a fluid are due to the chemical groups, and their charged state. In the following sections, whenever chemistry is relevant, I will mention its role in creating the entity that we then go on to describe insofar as is possible with physics alone. For example, I will acknowledge the charge of a nanopore is due to chemical groups attached to its surface, but will then proceed agnostically, treating the charges as if they were a perfect model surface-charge, uniformly distributed over the surface of the pore.

\begin{figure}[H]
	\includegraphics{/home/prestonh/Desktop/Researchscales.png
\end{figure}


\subsection{The macroscale---Navier-Stokes equations}












\end{document}
